%--------------------------------------
%   Header
%--------------------------------------
\documentclass[
    appendixprefix=true,
    a4paper,                    % Papierformat A4
    10pt,                       % Schriftgröße 10pt
    headings=normal,            % kleinere Überschriften verwenden
    chapterprefix=false,        % Einfügen von "Anhang" bzw. "Kapitel" in Überschrift
    oneside,                    % oneside - twoside
    openright,                  % einseitiges Layout
    titlepage,                  % Titleseite verwenden
    listof=totoc,               % alle Listen in das Inhaltsverzeichnis
    headsepline,                % Trennlinie zum Seitenkopf Bereich headings
    plainheadsepline,           % Trennlinie zum Seitenkopf Bereich Plain
    bibliography=totoc,         % das Literaturverzeichnis in den TOC
    parskip=half-,              % Abstand nach Absatz
    numbers=noenddot,           % damit hinter der letzten Ziffer kein Punkt steht (Kapitelnummerierung)
%draft                      % Draft Modus - Zeigt überschreitungen vom Seitenrand und rendert das Dokument
% schneller, da Bilder nur als Dummy dargestellt werden.
]{scrreprt}

%%--------------------------------------
%%   Packages
%%--------------------------------------
\usepackage[utf8]{inputenc}        % Zeichenkodierung - !!!Achtung, alle Dateien auch im UTF8 speichern!!!
\usepackage[T1]{fontenc}            % westeuropäische Schriftzeichenkodierung
\usepackage[ngerman]{babel}        % Babel-System
\usepackage[left=4cm, right=3cm, top=2.5cm, bottom=2.5cm]{geometry} % Seitenränder
\usepackage{array}                    % erweiterte Tabelleneigenschaften
\usepackage{graphicx}                % Grafiken
\usepackage{subfigure}                % Grafiken nebeneinander mit (a) und (b)
\usepackage[dvipsnames]{xcolor}
\usepackage{listings}
\usepackage{xspace}
\usepackage{ragged2e}
%\usepackage{hyperref}

\renewcommand{\labelenumi}{\arabic{enumi}.}
\renewcommand{\labelenumii}{\arabic{enumi}.\arabic {enumii}}
\renewcommand{\labelenumiii}{\arabic{enumi}.\arabi c{enumii}.\arabic{enumiii}}

%%--------------------------------------
%%   BibLateX - Literaturverzeichnis
%%--------------------------------------
\usepackage[style=authoryear]{biblatex}
\addbibresource{../bib/literatur.bib}


%---------------------------------------
%  Befehl für Schriftart Helvet / Arial
%---------------------------------------
\usepackage{helvet}
\renewcommand{\familydefault}{\sfdefault}
\usepackage[onehalfspacing]{setspace}    % 1,5 facher zeilen bastand aber nur im text nicht in Fußnoten oder verzeichnissen
\usepackage{amssymb}        % Mathesymbole
\usepackage{amsfonts}        % mathematische Schriftarten
\usepackage{amsmath}        % Mathepaket
\usepackage{cancel}            % Durchstreichungen wie beim kürzen
\usepackage{mathcomp}        % weitere Symbole
\usepackage{scrhack}        % verbessert einige Fremdklassen in Zusammenspiel mit KomaScript
\usepackage[babel,german=quotes]{csquotes}
\usepackage[ngerman]{translator}
\usepackage{epstopdf}
\usepackage{tikz}            % Für selbsterstellte Graphiken
\usepackage{booktabs}        % Für schönere Tabellen

\usepackage{pdfpages}


%smaller spaces in lists
\usepackage{enumitem}
\setitemize{noitemsep,topsep=0pt,parsep=0pt,partopsep=0pt}
\setenumerate{noitemsep,topsep=0pt,parsep=0pt,partopsep=0pt}

\setlist[itemize, 2]{label=\circ}

%%--------------------------------------
%%   Glossar
%%--------------------------------------
\usepackage[toc,acronym,nonumberlist]{glossaries}
\makeglossaries


%--------------------------------------
%   PDF Lesezeichen und Hyperlinks
%--------------------------------------
\usepackage[
pdfauthor={\autor},
pdftitle={{\titel { - }\untertitel}},
pdfsubject={{\titel { - }\untertitel}},
pdfkeywords={\keywords},
pdfpagelabels = {true},
pdfstartview = {FitV},
colorlinks = {true},
linkcolor = {black},
citecolor = {black},
urlcolor = {blue},
bookmarksopen = {true},
bookmarksopenlevel = {3},
bookmarksnumbered = {true},
plainpages = {false},
hypertexnames = {false}
]{hyperref}


%--------------------------------------
%   Kopf- & Fußzeile
%--------------------------------------
\usepackage[
    automark,
    plainheadsepline,
    autooneside,
    draft=false
]{scrlayer-scrpage}

\pagestyle{scrheadings}
\clearpairofpagestyles                                  % Kopf und Fuzeile lschen
\ihead[\hochschule]{\hochschule}                        % im Kopf -> links
%\chead{\headmark}                                       % im Kopf -> mitte
\ohead[\fachgebiet]{\fachgebiet}                        % im Kopf -> rechts
\cfoot[\pagemark]{\pagemark}                            % Seitenzahl -cfoot für center - ofoot für outer
\renewcommand*{\headfont}{\upshape\sffamily\scriptsize} % Schrift Kopfzeile
\renewcommand*{\footfont}{\normalfont\sffamily\small}   % Schrift Fußzeile


%--------------------------------------
%   Quellcode-Listing Einstellungen
%   ftp://ftp.tu-chemnitz.de/pub/tex/macros/latex/contrib/listings/listings.pdf
%--------------------------------------
\lstdefinelanguage{kotlin}{
    comment=[l]{//},
    commentstyle={\color{gray}\ttfamily},
    emph={filter, first, firstOrNull, forEach, lazy, map, mapNotNull, println},
    emphstyle={\color{OrangeRed}},
    identifierstyle=\color{black},
    keywords={
        !in, !is, abstract, actual, annotation, as, as?, break, by, catch, class, companion, const, constructor,
        continue, crossinline, data, delegate, do, dynamic, else, enum, expect, external, false, field, file, final,
        finally, for, fun, get, if, import, in, infix, init, inline, inner, interface, internal, is, lateinit, noinline,
        null, object, open, operator, out, override, package, param, private, property, protected, public, receiveris,
        reified, return, return@, sealed, set, setparam, super, suspend, tailrec, this, throw, true, try, typealias,
        typeof, val, var, vararg, when, where, while
    },
    keywordstyle={\color{NavyBlue}\bfseries},
    morecomment=[s]{/*}{*/},
    morestring=[b]",
    morestring=[s]{"""*}{*"""},
    ndkeywords={
        @Deprecated, @JvmField, @JvmName, @JvmOverloads, @JvmStatic, @JvmSynthetic, Array, Byte, Double, Float,
        Int, Integer, Iterable, Long, Runnable, Short, String
    },
    ndkeywordstyle={\color{BurntOrange}\bfseries},
    sensitive=true,
    stringstyle={\color{ForestGreen}\ttfamily},
    showstringspaces=false,
    numbers=left,                               %Zeilennummerierung links
    frame=single,                               %Einfacher Rahmen
    breaklines=true,                            %Automatische Zeilenumbrüche
    numberstyle=\small,                         %Größe der Zeilennummerierung
    basicstyle={\small} ,                       %Schriftgröße
    backgroundcolor=\color{Gray!3},               %Hintergrundfarbe
    captionpos=t,                               %Überschift oben (top)
}
\usepackage{listings}

\usepackage{color}
\definecolor{gray}{rgb}{0.4,0.4,0.4}
\definecolor{darkblue}{rgb}{0.0,0.0,0.6}
\definecolor{cyan}{rgb}{0.0,0.6,0.6}
\definecolor{darkgreen}{rgb}{0,0.5,0}

\lstset{
    basicstyle=\ttfamily,
    columns=fullflexible,
    showstringspaces=false,
    commentstyle=\color{gray}\upshape
}

\lstdefinelanguage{XML}
{
    morestring=[b]",
    morestring=[s]{>}{<},
    morecomment=[s]{<?}{?>},
    stringstyle=\color{darkgreen},
    identifierstyle=\color{darkblue},
    keywordstyle=\color{cyan},
    morekeywords={xmlns,version,type, android},% list your attributes here
    numbers=left,                               %Zeilennummerierung links
    frame=single,                               %Einfacher Rahmen
    breaklines=true,                            %Automatische Zeilenumbrüche
    numberstyle=\small,                         %Größe der Zeilennummerierung
    basicstyle={\small} ,                       %Schriftgröße
    backgroundcolor=\color{Gray!3},               %Hintergrundfarbe
    captionpos=t                               %Überschift oben (top)
}
\lstdefinelanguage{cpp}{                            %hier Sprache einstellen
    basicstyle={\small} ,                           %Schriftgröße
    keywordstyle=\color{blue!80!black!100},         %Farbe der keywords
    identifierstyle=,                               %Bezeichnerstyle, hier leer
    commentstyle=\color{green!50!black!100},        %Farbe der Kommentare
    stringstyle=\ttfamily,                          %Aussehen der Strings
    breaklines=true,                                %Automatische Zeilenumbrüche
    numbers=left,                                   %Zeilennummerierung links
    numberstyle=\small,                             %Größe der Zeilennummerierung
    frame=single,                                   %einfacher Rahmen
    backgroundcolor=\color{blue!3},                 %Hintergrundfarbe
    caption={Code},                                 %Standardüberschrift
    captionpos=t,                                   %Überschift oben (top)
    literate=                                       %UTF8
        {Ä}{{\"A}}1
        {Ö}{{\"O}}1
        {Ü}{{\"U}}1
        {ß}{{\ss}}1
        {ä}{{\"a}}1
        {ö}{{\"o}}1
        {ü}{{\"u}}1
        {~}{{\textasciitilde}}1
}
\definecolor{dkgreen}{rgb}{0,0.6,0}
\definecolor{ltgray}{rgb}{0.5,0.5,0.5}

\usepackage{listings}
\lstset{%
  	breakatwhitespace=false,
  	commentstyle=\color{dkgreen},
  	deletekeywords={...},
  	escapeinside={\%*}{*)},
  	extendedchars=true,
  	keepspaces=true,
  	keywordstyle=\color{blue},
  	language=SQL,
  	morekeywords={*,modify,MODIFY,...},
  	numbersep=15pt,
  	rulecolor=\color{ltgray},
  	showspaces=false,
  	showstringspaces=false,
  	showtabs=true,
  	stepnumber=1,
  	tabsize=4,
  	title=\lstname,
	numbers=left,                               %Zeilennummerierung links
	frame=single,                               %Einfacher Rahmen
	breaklines=true,                            %Automatische Zeilenumbrüche
	numberstyle=\small,                         %Größe der Zeilennummerierung
	basicstyle={\small} ,                       %Schriftgröße
	backgroundcolor=\color{Gray!3},               %Hintergrundfarbe
	captionpos=t                               %Überschift oben (top)
}
\lstset{literate=%
        {Ö}{{\"O}}1
        {Ä}{{\"A}}1
        {Ü}{{\"U}}1
        {ß}{{\ss}}1
        {ü}{{\"u}}1
        {ä}{{\"a}}1
        {ö}{{\"o}}1
}

%----------Worttrennung-----------
%---Latex trennt eigentlich recht gut, aber Fremdwörter o.ä. manchmal nicht, daher kann man Latex das Trennen einzelner Wörter beibringen, z.B.:
\hyphenation{Ein-gangs-pro-zess}
\hyphenation{Aus-gangs-pro-zess}
\hyphenation{Web-client}
\hyphenation{DMS-Web-client}
\hyphenation{Wire-frame}
% here you can create all keywords from your thesis
% for examble you can create a shortcut for the keywords "zum Beispiel"
% for this write  \newcommand{\zb}{zum Beispiel }
% !! dont forget the \xspace at the end

% \xspace provieds a space, when there is other text behind
% if there is something like a comma (,) then there will be no space
\newcommand{\keywordEins}{Keyword 1\xspace}
\newcommand{\keywordZwei}{Keyword 2\xspace}
\newcommand{\keywordDrei}{Keyword 3\xspace}
\newcommand{\keywordVier}{Keyword 4\xspace}}
%--------------------------------------
%   Eigene Funktionen
%--------------------------------------
\newcommand{\quoteWithFootCite}[3]{
    \begingroup\linespread{0.75}\selectfont
    \begin{quote}
        \begin{center}
            \glqq #1 \grqq{}
        \end{center}
        \begin{flushright}
            \textbf{#2} ~ \footcite{#3}
        \end{flushright}
    \end {quote}
    \endgroup
}

\newcommand{\quotations}[1]{
    \glqq #1\grqq{}
}


%-----reference-----
%(siehe Anhang B.4 Zeile 3 - 7)
\newcommand{\refAtt}[1]{(siehe Anhang #1\xspace)}

%(siehe Abschnitt 5.5.4 auf Seite 37)
\newcommand{\refSec}[1]{(siehe Abschnitt #1\xspace)}

%(siehe Listing 5.4 Zeile 33 - 37)
\newcommand{\refLst}[1]{(siehe Listing #1\xspace)}



\newcommand{\catchphrase}[1]{\textit{#1}}

\definecolor{codeGray}{RGB}{240,240,240}
\newcommand{\inlineCode}[1]{
    \colorbox{codeGray}{\textit{\lstinline|#1|}}
}

\newlist{tableitemize}{itemize}{2}
\setlist[tableitemize, 1]{leftmargin=*, after=\vskip-\baselineskip,before=\vskip-\baselineskip, nosep,label=$\bullet$}
\setlist[tableitemize, 2]{leftmargin=*, after=\vskip-\baselineskip,before=\minipagetrue, nosep,label=\circ}

\newlist{tableenumerate}{enumerate}{2}
\setlist[tableenumerate, 1]{leftmargin=*,nosep,label*=\arabic*.}
\setlist[tableenumerate, 2]{leftmargin=*,nosep,label*=\arabic*.}


\makeindex

\newcommand{\art}{Bachelorarbeit}
\newcommand{\titel}{Namer deiner BA mit Keywords wie: \keywordEins und \keywordZwei}
%%\newcommand{\untertitel}{Exposé}
\newcommand{\autor}{Vorname Nachname}
\newcommand{\registriernr}{Nr. AI-20XX-BA-XX}
\newcommand{\hochschule}{Fachhochschule Erfurt}
\newcommand{\fachgebiet}{Angewandte Informatik}
\newcommand{\erstgutachter}{Prof. Dr. Vorname Nachname}
\newcommand{\zweitgutachter}{Prof. Dr. Vorname Nachname}
\newcommand{\datum}{xx.xx.20xx} % \today gibt heutiges Datum aus, oder klassisch durch Abgabedatum zu ersetzen
\newcommand{\ort}{Erfurt}


\newglossaryentry{TestCode}{
    name={TestName2},
    description={TestBeschreibung3}
}









