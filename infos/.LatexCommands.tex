%--------------------------------ALLGEMEINE FORMATIERUNG-------------------------------------
\textit{}           % Kursiv
\textbf{} \newline  % Bold
\glqq Text\grqq{}   % Anführungszeichen
\quotations{}       % Anführungszeichen in einem Befehl
\catchphrase{}      %Schlagwörter
%--------------------------------ZITIEREN-------------------------------------
\autocite { Toussaint 2013}                 %   (Toussaint 2013)
\autocite [17]{ Toussaint 2013}             %   (Toussaint 2013, S. 17)
\autocite [ vgl .][17]{ Toussaint 2013}     %   (vgl. Toussaint 2013, S. 17)
\autocite [ vgl .][]{ Toussaint 2013}       %   (vgl. Toussaint 2013)
\cite { Toussaint 2013}                     %   Toussaint 2013
\footcite { Toussaint 2013}                 %   1 - 1Toussaint 2013.
\fullcite { Toussaint 2013}                 %   Godfried T. Toussaint (2013). The Geometry of Musical Rhythm: What Makes a”good“ Rhythm Good? Boca Raton, Fla.:CRC Press, Taylor & Francis Group. isbn:978-1-4665-1202-3
\parencite *{ Toussaint 2013}               %   (2013)

\begin{center} //Zitat Formatiert
    \textit{\glqq ZITAT \grqq{}}
\end{center}

\quoteWithFootCite{Shared UI is a history of pain and failure. Shared logic is the history of computers.}{GalliganHistory}

%--------------------------------REFERENZIERUNG-------------------------------------
~\ref{fig:fh_logo}
\ref{lst:testVerlinkung}
\pageref{fig:test}
\refAtt{\ref{lst:build_src_common} Zeile 20 - 55}
(siehe Kapitel \ref{ch:CHAPTER} Abschnitt \ref{sec:SECTION} auf Seite \pageref{sec:SECTION})


%   ch:	    chapter
%   sec:	section
%   subsec:	subsection
%   fig:	figure
%   tab:	table
%   eq:	    equation
%   lst:	code listing
%   itm:	enumerated list item
%   alg:	algorithm
%   app:	appendix subsection

%--------------------------------CODE-------------------------------------
\begin{lstlisting}[caption={REPLACECODE},label={lst:replaceCode}, language=kotlin]
CODE
\end{lstlisting}

// Excluded from List of Listings
\begin{lstlisting}[caption={REPLACECODE},nolol,label={lst:replaceCode}, language=kotlin]
CODE
\end{lstlisting}


%--------------------------------BILDER-------------------------------------
% 2 Bilder nebeneinander
\begin{figure}[!h]
    \centering
    \begin{minipage}[b]{.4\linewidth} % [b] => Ausrichtung an \caption
        \includegraphics[scale=0.5]{images/replace}\captionof{figure}{REPLACE}
        \label{fig:replace}
    \end{minipage}
    \hspace{.1\linewidth}% Abstand zwischen Bilder
    \begin{minipage}[b]{.4\linewidth} % [b] => Ausrichtung an \caption
        \includegraphics[scale=0.5]{images/replace}\captionof{figure}{REPLACE}
        \label{fig:replace}
    \end{minipage}
\end{figure}


%Bild scaled
\begin{center}
    \includegraphics[scale=0.4]{images/replace}
    \captionof{figure}{REPLACE}
    \label{fig:replace}
\end{center}

%Bild Volle Breite
\begin{center}
    \includegraphics[width=\columnwidth]{images/replace}
    \captionof{figure}{Architektur}
    \label{fig:replace}
\end{center}


%--------------------------------LISTEN-------------------------------------
\begin{itemize}
    \item
    \item
    \item
    \item
    \item
    \item
\end{itemize}


%--------------------------------TABELLEN-------------------------------------
\begin{tabular}{|p{2cm}|p{3cm}|r|l|}
                                                                                                                        \hline
                                    & Android                   & \ios                   &  Shared                   \\  [0.5ex]  \hline \hline
    \ui                  & JP Compose                & Swift UI              &                           \\ \hline
    Images                          & Coil                      & SDWebImageSwiftUI     &                           \\ \hline
    Navigation                      & JP Compose Navigation     & Swift UI              &                           \\ \hline
    Dependency Management           & Gradle                    & Cocoapods             &  Gradle                   \\ \hline
    Dependency Injection            &                           &                       &    Koin                   \\ \hline
    Serialization                   &                           &                       &  KotlinX Serialization    \\ \hline
    Datenbank                       &                           &                       &  SQL Delight              \\ \hline
    Remotezugriff                   &                           &                       &  Ktor                     \\ \hline
    Datumsformat                    &                           &                       &  KotlinX Datetime         \\ \hline
    Unit Tests                      &                           &                       &  Kotlin Test              \\ \hline
    UI-Test                         & JP Compose                & XCodeTests            &                           \\ \hline
    Asynchronität                   & KotlinX Coroutines        & Core Foundation*      &  KotlinX Coroutines       \\ \hline
\end{tabular}





%--------------------------------GLOSSAR-------------------------------------
Description of commands used in above example:

\gls{<label>}
% This command prints the term associated with <label> passed as its argument. If the hyperref package was loaded
% before glossaries it will also be hyperlinked to the entry in glossary.

\glspl{<label>}
% This command prints the plural of the defined term, other than that it behaves in the same way as gls.

\Gls{<label>}
% This command prints the singular form of the term with the first character converted to upper case.

\Glspl{<label>}
% This command prints the plural form with first letter of the term converted to upper case.

\glslink{<label>}{<alternate text>}
% This command creates the link as usual, but typesets the alternate text instead. It can also take several options
% which changes its default behavior (see the documentation).

\glssymbol{<label>}
% This command prints what ever is defined in \newglossaryentry{<label>}{symbol={Output of glssymbol}, ...}

\glsdesc{<label>}
% This command prints what ever is defined in \newglossaryentry{<label>}{description={Output of glsdesc}, ...}

